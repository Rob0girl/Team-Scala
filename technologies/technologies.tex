
\section{Technologies}
The technologies to be used for the system will be discussed below:
\begin{itemize}
	\item TOMCAT - it is a free Apache application server where one can deploy and run JSP and Servlets and provides good logging and management facilities since every app gets wrapped into a war file and is mostly used for web applications as its limited to functionality.
	
	\item WILDFLY - similar to tomcat server but this is a full fledged JavaEE application server for the back-end system.
	
	\item JAVAEE - will be used develop the business logic of NavUP which forms part of the back-end. JavaEE is a good choice for NavUP because this platform uses "containers" to simplify development. JavaEE containers provide for the separation of business logic from resource and life-cycle management, which means that developers can focus on writing business logic, rather than writing enterprise infrastructure. 
	
	\item TCP/IP - the reason why we chose this over UDP since we want to know if the user has received the data we sent which will help making sure that the user has the correct map all the time and since it's end-to-end we can also have encryption for safety and privacy issues.
	
	\item JAX-RS REST - Using the RESTEasy implementation of JAX-RS since it is portable and can run on any Servlet container, and integrates nicely with JBOSS and provides client framework to make writing HTTP clients easy.
	
	\item ACTIVEMQ - is a message broker which deals with synchronization and messaging queue facilities which will be necessary for notifications module, it provides a lot of features such as REST API to provide  technology agnostic and language neutral web based API to messaging, supports Ajax, fast and supports pluggable protocols such as TCP and SSL in specific that we will use. 
	
	\item JPA - The Java Persistence API (JPA) is a specification for object-relational mapping in Java. The main advantage JPA will have on NavUP will be its database independence. JPA It provides a database independent abstraction on top of SQL. As long as you're not using any native queries, you don't have to worry about database portability. 
	
	\item POSTGRESQL / POSTGIS - The entire code of the system should be decoupled from which concrete database is used. For NavUP PostgreSQL will be used. It is a mature, efficient and reliable relational database implementation which is available across platforms and for which there is a large and very competent support community
	
	\item JUnit - JUnit will used to perform unit tests on separate modules of NavUP. Junit has been chosen due to its simplicity when testing Java applications. JUnit can be used separately or integrated with build tools like Maven and Ant and third party extensions.
	
	\item JSP - Java Server Pages (JSP) is a technology that helps software developers create dynamically generated web pages based on HTML, XML, or other document types. The main benefit that JSP will have on NavUP is that JSP pages are compiled dynamically into Servlets when requested, so page authors can easily make updates to presentation code.
	
	\item HTML5 \& BOOTSTRAP - The main advantage of using bootstrap on NavUP will be its responsiveness.Creating mobile ready websites is a breeze with Bootstrap thanks to the fluid grid layout that dynamically adjusts to the proper screen resolution. There is virtually no work that needs to be done to achieve proper responsiveness. HTML5 is a markup language used for structuring and presenting content on the internet. One of the coolest things about HTML5 is the new local storage feature which is better than cookies because it allows for storage across multiple windows and also has better security and performance and data will persist even after the browser is closed.
	
	\item JQUERY - works off JavaScript providing easier methods and powerful functionality to developers with ease, it provides easier methods such as Ajax which makes it better for communications, and event handling for triggering functionality when certain conditions are met.  
	
	\item CSS - CSS(Cascading Style Sheet) will be used for styling the web front end of NavUP. It was chosen because of its Lightweight code and because it's easier to maintain and update. 
	
	\item MAVEN - Maven is a powerful project management tool that is based on POM(Project Object Model).It is used for projects build,dependency and documentation. Maven will be suitable for NavUP because maven projects do not need to store third-party binary libraries in source control, which reduces undue stress on developer checkouts and builds which results in better dependency management.
	
	\item POSITION LOGIC - is a GPS tracking platform that provides accurate information about the users location which we can incorporate to make the user experience and accurateness of the map more significant.
	
	
	\item IONIC v2 - is a open source hybrid mobile application framework which deals with web technologies such as HTML, Angular, JavaScript and Typescript to make a nice looking application with minimal effort, they have their own custom styling to make it easier to build native looking applications with ease. It is also powerful for data processing since v2 works with typescript which is a OOP language derived from JavaScript, this typescript gets transpiled into JavaScript at run time. It unlocks native APIs and features by wrapping Cordova plug-ins into typescript especially good for accessing wireless information and also GPS.
	
	
	
\end{itemize}
