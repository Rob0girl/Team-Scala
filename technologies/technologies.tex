
\section{Technologies}
The technologies to be used for the system will be discussed below:
\begin{itemize}
	\item TOMCAT - it is a free apache application server where one can deploy and run JSP and Servlets and provides good logging and management facilities since every app gets wrapped into a war file and is mostly used for web applications as its limited to functionality.
	
	\item WILDFLY - similar to tomcat server but this is a full fledged JavaEE application server for the back-end system.
	
	\item JAVAEE - will be used develop the business logic of NavUP which forms part of the back-end. JavaEE is a good choice for NavUP because this platform uses "containers" to simplify development. JavaEE containers provide for the separation of business logic from resource and lifecycle management, which means that developers can focus on writing business logic, rather than writing enterprise infrastructure. 
	
	\item TCP/IP - 
	
	\item JAX-RS REST - Using the RESTEasy implementation of JAX-RS since it is portable and can run on any servlet container, and integrates nicely with JBOSS and provides client framework to make writing HTTP clients easy.
	
	\item ACTIVEMQ - 
	
	\item JPA - The Java Persistence API (JPA) is a specification for object-relational mapping in Java. The main advantage JPA will have on NavUP will be its database independence. JPA It provides a database independent abstraction on top of SQL. As long as you're not using any native queries, you don't have to worry about database portability. 
	
	\item POSTGRESQL / POSTGIS - 
	
	\item JUnit - JUnit will used to perform unit tests on separate modules of NavUP. Junit has been choosen due to its simplicity when testing java applications. JUnit can be used separately or integrated with build tools like Maven and Ant and third party extensions
	
	\item JSP - 
	
	\item HTML5 \& BOOTSTRAP - The main advantage of using bootstrap on NavUP will be its responsiveness.Creating mobile ready websites is a breeze with Bootstrap thanks to the fluid grid layout that dynamically adjusts to the proper screen resolution. There is virtually no work that needs to be done to achieve proper responsiveness.
	
	\item JQUERY - 
	
	\item CSS - CSS(Cascading Style Sheet) will be used for styling the web front end of NavUP. It was chosen because of its Lightweight code and because it's easier to maintain and update. 
	
	\item MAVEN - 
	
	
	
	\item IONIC v2 - 
	
	\item CORDOVA - 
	
	
	
\end{itemize}
