
\section{Technologies}
The technologies to be used for the system will be discussed below:
\begin{itemize}
	\item TOMCAT - 
	
	\item JAVAEE - JavaEE will be used develop the business logic of NavUp which forms part of the backend.JavaEE is a good choice for NavUp because this platform uses "containers" to simplify development. JavaEE containers provide for the separation of business logic from resource and lifecycle management, which means that developers can focus on writing business logic, rather than writing enterprise infrastructure. 
	
	\item TCP/IP - 
	
	\item RESTEASY - 
	
	\item ACTIVEMQ - 
	
	\item JPA - The Java Persistence API (JPA) is a specification for object-relational mapping in Java. The main advantage JPA on NavUP will be its database independence. JPA It provides a database independent abstraction on top of SQL.As long as you’re not using any native queries, you don’t have to worry about database portability. 
	
	\item POSTGRESQL / POSTGIS- 
	
	
	
	\item JSP - 
	
	\item BOOTSTRAP - The main advantage of using bootstrap on NavUp will be its responsiveness.Creating mobile ready websites is a breeze with Bootstrap thanks to the fluid grid layout that dynamically adjusts to the proper screen resolution. There is virtually no work that needs to be done to achieve proper responsiveness.
	
	\item JQUERY - 
	
	\item CSS - CSS(Cascading Style Sheet) will be used for styling the web front end of NavUP. It was chosen because of its Lightweight code and because it's easier to maintain and update. 
	
	\item MAVEN - 
	
	
	
	\item IONIC v2 - 
	
	\item CORDOVA - 
	
	
	
\end{itemize}
